\begin{lstlisting}[caption={Estimation Parameters.}, label={lst:est_param}]
%% ESTIMATION PARAMETERS

values = meas.signals.values;
time = meas.time;

values_trunc = values(5400:7000);                      

values_mean = movmean(values_trunc,9);              

[peaks,locs] = findpeaks(values_mean);

locs_new = [];
peaks_new = [];

for ii=1:length(locs)
    if(ii+1 <= length(locs))
        if(locs(ii)+80<locs(ii+1))                  
            locs_new = [locs_new,locs(ii)];
            peaks_new = [peaks_new,peaks(ii)];
        end
    else
        locs_new = [locs_new,locs(ii)];
        peaks_new = [peaks_new,peaks(ii)];
    end
end

figure;
plot(values_mean);
hold on;
scatter(locs_new,peaks_new);
hold off;

%% REGRESSION

M = length(locs_new);                               % Number of peaks

Y = log(peaks_new).';

phi = [];
for k = 1:M
    phi = [phi; -k, 1];
end
             
theta_hat_LS = phi \ Y;                             % least square solution (theta_hat_LS = inv(phi.'*phi)*phi.'*Y;)

a_hat = theta_hat_LS(1);                
b_hat = theta_hat_LS(2);

csi_hat = a_hat;                                    % LS estimate of the logarithmic decrement

delta_hat = csi_hat/(sqrt(pi^2 + csi_hat^2));       % Estimate of the damping factor   

omega_hat = 0;
for k = 1:M-1                                        
    T_k = (locs_new(k+1) - locs_new(k)) / 1000;
    omega_hat_k = pi/T_k;
    omega_hat = omega_hat + omega_hat_k;
end

omega_hat = omega_hat/(M-1);                        

omega_hat_n = omega_hat/sqrt(1-delta_hat^2);        % Estimate of the natural frequency

sigma = - delta_hat*omega_hat_n;
omega = omega_hat_n*sqrt(1-delta_hat^2);

J_b = 1.4e-3;
B_b = J_b * (2*delta_hat * omega_hat_n);            %0.0046  %0.0032 LAB
k = J_b * omega_hat_n * omega_hat_n;                %0.8425  %0.8162 LAB
\end{lstlisting}