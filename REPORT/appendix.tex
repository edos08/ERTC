% -------------------------------------------------------------------------------- %
\appendix
\section{Appendix}



\subsection{Matlab code}
\begin{lstlisting}[caption={Estimation Parameters.}, label={lst:est_param}]
%% ESTIMATION PARAMETERS

values = meas.signals.values;
time = meas.time;

values_trunc = values(5400:7000);                      

values_mean = movmean(values_trunc,9);              

[peaks,locs] = findpeaks(values_mean);

locs_new = [];
peaks_new = [];

for ii=1:length(locs)
    if(ii+1 <= length(locs))
        if(locs(ii)+80<locs(ii+1))                  
            locs_new = [locs_new,locs(ii)];
            peaks_new = [peaks_new,peaks(ii)];
        end
    else
        locs_new = [locs_new,locs(ii)];
        peaks_new = [peaks_new,peaks(ii)];
    end
end

figure;
plot(values_mean);
hold on;
scatter(locs_new,peaks_new);
hold off;

%% REGRESSION

M = length(locs_new);                               % Number of peaks

Y = log(peaks_new).';

phi = [];
for k = 1:M
    phi = [phi; -k, 1];
end
             
theta_hat_LS = phi \ Y;                             % least square solution (theta_hat_LS = inv(phi.'*phi)*phi.'*Y;)

a_hat = theta_hat_LS(1);                
b_hat = theta_hat_LS(2);

csi_hat = a_hat;                                    % LS estimate of the logarithmic decrement

delta_hat = csi_hat/(sqrt(pi^2 + csi_hat^2));       % Estimate of the damping factor   

omega_hat = 0;
for k = 1:M-1                                        
    T_k = (locs_new(k+1) - locs_new(k)) / 1000;
    omega_hat_k = pi/T_k;
    omega_hat = omega_hat + omega_hat_k;
end

omega_hat = omega_hat/(M-1);                        

omega_hat_n = omega_hat/sqrt(1-delta_hat^2);        % Estimate of the natural frequency

sigma = - delta_hat*omega_hat_n;
omega = omega_hat_n*sqrt(1-delta_hat^2);

J_b = 1.4e-3;
B_b = J_b * (2*delta_hat * omega_hat_n);            %0.0046  %0.0032 LAB
k = J_b * omega_hat_n * omega_hat_n;                %0.8425  %0.8162 LAB
\end{lstlisting}

\begin{lstlisting}[caption={Pid Design Code.}, label={lst:pid_design}]
%% DESIGN OF PID CONTROLLER

ts_5 = 0.85;    % [s]
M_p = 0.3;      % 30%
alpha = 4;

delta = (log(1/M_p))/(sqrt(pi^2+log(1/M_p)*log(1/M_p)));

phi_m = atan((2*delta)/(sqrt(sqrt(1+4*(delta ^4)) - 2*(delta ^2))));

w_gc = 3/(delta*ts_5);

T_L = 1/(10*w_gc);
AW.T_W = ts_5/5;
AW.K_W = 4/AW.T_W;

D_tau = tf([Jeq*mld.Jb Jeq*mld.Bb+mld.Jb*Beq Beq*mld.Bb+mld.k*(Jeq+mld.Jb/gbox.N/gbox.N) mld.k*(Beq+mld.Bb/gbox.N/gbox.N)], 1);
sysPh_den = tf([mot.L mot.R+sens.curr.Rs], 1)*D_tau + mot.Kt*mot.Ke*tf([mld.Jb mld.Bb mld.k], 1);
sysPh = tf(drv.dcgain, [drv.Tc 1])*tf(1, [gbox.N 1])*tf([mot.Kt*mld.Jb mot.Kt*mld.Bb mot.Kt*mld.k], 1)*(1/sysPh_den);

z = freqresp(sysPh, w_gc);

mag = abs(z);
phi = angle(z);

d_k = 1/(abs(mag));
d_phi = -pi + phi_m - phi;

controller.Kp = d_k*cos(d_phi);
controller.Kd = controller.Kp*(tan(d_phi) + sqrt(tan(d_phi)^2 + 4/alpha))/(2*w_gc);
controller.Ki = controller.Kp^2/(alpha*controller.Kd);
\end{lstlisting}
\pagebreak

\begin{lstlisting}[caption={Nominal State Space Controller Design Code.},label={lst:statespace_design}]
%% Nominal State Space Controller Design

w_c = 2*50*pi;
delta_c = 1/sqrt(2);
ts_5 = 0.27;        % tuned for real world test
M_p = 0.3;

delta = (log(1/M_p))/(sqrt(pi^2+log(1/M_p)*log(1/M_p)));
w_n = 3/(ts_5*delta);

phi = atan(sqrt(1 - delta^2)/delta);

lam_c1 = w_n*exp(1i*(-pi + phi));
lam_c2 = conj(lam_c1);
lam_c3 = w_n*exp(1i*(-pi + phi/2));
lam_c4 = conj(lam_c3);

A = [0, 0, 1, 0; 
     0, 0, 0, 1; 
     0, mld.k/(gbox.N^2*Jeq), -(Beq + mot.Kt*mot.Ke/(mot.R + sens.curr.Rs))/Jeq, 0;
     0, -(mld.k/mld.Jb + mld.k/(Jeq*gbox.N^2)), -mld.Bb/mld.Jb+(Beq + mot.Kt*mot.Ke/(mot.R + sens.curr.Rs))/Jeq, -mld.Bb/mld.Jb];

B = [0; 0; (mot.Kt*drv.dcgain)/(gbox.N*Jeq*(mot.R + sens.curr.Rs)); -(mot.Kt*drv.dcgain)/(gbox.N*Jeq*(mot.R + sens.curr.Rs))];

B_d = [0; 0; -1/(gbox.N^2*Jeq); 1/(gbox.N^2*Jeq)];

C=[1, 0, 0, 0];

D = 0;

F = [A, B; C, 0];
sol = [0, 0, 0, 0, 1].';

N = linsolve(F, sol);
N_x = N(1:4);
N_u = N(5);

K = acker(A,B,[lam_c1, lam_c2, lam_c3, lam_c4]);

t_0 = 0.2;
t_1 = 0.7;
\end{lstlisting}
\begin{lstlisting}[caption={Nominal State Space Controller with Integral Action Design Code.},label={lst:statespace_robust_design}]
%% Nominal State Space Controller with Integral Action Design

A_e = [0, C; [0; 0; 0; 0], A];
B_e = [0; B];
C_e = [0, C];
D_e = 0;

K_e = acker(A_e, B_e, [lam_c1, lam_c2, lam_c3, lam_c4, lam_c5]);
K_I = K_e(1);
K_new = K_e(2:5);

t_0 = 0.2;
t_1 = 0.7;
\end{lstlisting}
\pagebreak

\begin{lstlisting}[caption={LQR obtained using argumentations based on SRL.}, label={lst:lqr-srl}]
%% LQR obtained using argumentations based on Simmetric Root Locus (SRL)

sysG = ss(A, B, C, 0);
sysGp = ss(-A, -B, C, 0);

sigma = delta*w_n;

[roots,g] = rlocus(sysG*sysGp);

gain = 2.3e3;           % Value to be tuned based on root locus plot
r = 1/gain;       

poles = rlocus(sysG*sysGp,1/gain);

% Simmetric Root Locus Plot
figure;
rlocus(sysG*sysGp);
axis([-60,60,-60,60]);
hold on;
plot(-g,exp(phi)*g);
plot(-g,-g*exp(phi));
plot(real(poles), imag(poles), 'rx','Markersize', 10);
xline(sigma);
hold off;

% State Feedback Gain Computation
K = lqry(sysG,1,gain);
\end{lstlisting}
\begin{lstlisting}[caption={LQR obtained using the Bryson’s rule.}, label={lst:lqr-br}]
%% LQR obtained using the Bryson's rule

sysP = ss(A, B, C, 0);

Q = diag([1/(0.3*50*pi/180)^2 , 1/(pi/36)^2, 0, 0]);
R = 1/100;

% State Feedback Gain Computation
K_br = lqr(sysP, Q ,R);
\end{lstlisting}
\begin{lstlisting}[caption={Robust tracking with LQR obtained using argumentations based on SRL.}, label={lst:lqr-srl-robust}]
%% Robust tracking with LQR obtained using argumentations based on Simmetric Root Locus (SRL)

sysG = ss(A_e, B_e, C_e, 0);
sysGp = ss(-A_e, -B_e, C_e, 0);

sigma = delta*w_n;

[roots, g] = rlocus(sysG*sysGp);

gain = 1.62e03;       % Value to be tuned based on root locus plot
r = 1/gain;

poles = rlocus(sysG*sysGp, 1/gain);

% Simmetric Root Locus Plot
figure;
rlocus(sysG*sysGp);
axis([-60,60, -60,60]);
hold on;
plot(-g, g*exp(phi));
plot(-g,-g*exp(phi));
plot(real(poles), imag(poles), 'rx','Markersize', 10);
xline(sigma);
hold off;

% Extended State Feedback Gain Computation
K_e = lqry(sysG, 1, gain);
K_I = K_e(1);
K_new = [K_e(2), K_e(3), K_e(4), K_e(5)];
\end{lstlisting}
\begin{lstlisting}[caption={Robust tracking with LQR obtained using the Bryson's rule.}, label={lst:lqr-br-robust}]
%% Robust tracking with LQR obtained using the Bryson's rule

sysP = ss(A_e, B_e, C_e, 0);

q_11 = 10^(3);
Q = diag([q_11, 1/(0.3*50*pi/180)^2 , 1/(pi/36)^2, 0, 0])
R = 1/100;

% Extended State Feedback Gain Computation
K_e_br = lqr(sysP, Q, R)
K_I = K_e_br(1);
K_new = K_e_br(2:5);
\end{lstlisting}
\begin{lstlisting}[caption={Frequency Shaped LQR for Nominal Tracking.}, label={lst:fs-resonant-freq}]
%% Frequency Shaped LQR for Nominal Tracking

sysP = ss(A, B, C, 0);

sys_poles = pole(sysP);

s = tf('s');

den = 1;
for index = 1:height(sys_poles)
    if (imag(sys_poles(index)) ~= 0)
        den = den * (s - sys_poles(index));
    end
end

den_array = cell2mat(den.Numerator);
w_0 = sqrt(den_array(1, end));

q_22 = 100;         % {0.01, 1, 100}

A_1_Q = [0, 1; -w_0^2, 0];
B_1_Q = [0; 1];
C_1_Q = [sqrt(q_22)*w_0^2, 0];
D_1_Q = 0;

A_Q = A_1_Q;
B_Q = [[0; 0], B_1_Q, [0, 0; 0, 0]];
C_Q = [[0, 0]; C_1_Q; [0, 0; 0, 0]];
D_Q = diag([1/(5*pi/180), D_1_Q, 0, 0]);

A_A = [A, [0, 0; 0, 0; 0, 0; 0, 0]; B_Q, A_Q];
B_A = [B; [0; 0;]];
C_A = [C, 0, 0];
D_A = 0;

Q_A = [D_Q.'*D_Q, D_Q.'*C_Q; C_Q.'*D_Q, C_Q.'*C_Q];
N_A = 0;
R_A = 1/(uh_bar^2);

% State Feedback Gain Computation
K_A = lqr(A_A, B_A, Q_A, 0.01, 0);

% Feedforward Gains Computation
F = [A_A, B_A; C_A, D_A];
sol = [0, 0, 0, 0, 0, 0, 1].';

N = linsolve(F, sol);
N_x = N(1:6);
N_u = N(7);
\end{lstlisting}

\begin{lstlisting}[caption={Frequency Shaped LQR with Integral Action.}, label={lst:fs-resonant-freq-robust}]
%% Frequency Shaped LQR with Integral Action

A_e = [0, C_A; zeros(6, 1), A_A];
B_e = [0; B_A];
C_e = [0, C_A];

q_I = 0.1;

Q_e = [q_I, zeros(1, 6); zeros(6, 1), Q_A];
R_e = R_A;

% Extended State Feedback Gain Computation
K_e = lqr(A_e, B_e, Q_e, R_e);
K_I = K_e(1);
K_new = K_e(2:7);
\end{lstlisting}
\pagebreak
\subsection{Additional computations for the estimation of parameters} \label{estimation}
Given the set of measurements \(Z^M=\{t_k , |\vartheta_d(t_k)|\}\) with \(k = 0,1,...,M-1\), which consists of \(M\) consecutive peaks of the absolute value of the natural response, it is possible to introduce the following notation
\begin{equation}
    z = \boldsymbol{\varphi}_k^T\boldsymbol{\theta}
\end{equation}
where
 \begin{equation}
    \label{eq : vectors}
        \boldsymbol{\varphi}_{k}^T = [-k, 1], \qquad 
        \boldsymbol{\theta} = [a, b]^T.
    \end{equation}
Then, the LS fitting can be determined by finding the value of \(\boldsymbol{\theta}\) that minimises the quadratic error
\begin{equation}
\label{eq:cost}
    V(\boldsymbol{\theta}) = \sum_{k=0}^{M-1} (log|\vartheta_d(t_k)|-\boldsymbol{\varphi}_{k}^T \boldsymbol{\theta})^2
\end{equation}
\\Using the notation
\begin{equation}
    \boldsymbol{Y} =
    \begin{bmatrix} log|\vartheta_d(t_0)| \\ log|\vartheta_d(t_1)|\\ \vdots \\ log|\vartheta_d(t_{M-1})|
    \end{bmatrix}
    \in \mathbb{R}^{M\times1},\qquad
    \boldsymbol{\Phi} = 
    \begin{bmatrix} \boldsymbol{\varphi}_{0}^T \\ \boldsymbol{\varphi}_{1}^T \\ \vdots \\ \boldsymbol{\varphi}_{M-1}^T  
    \end{bmatrix}
    \in \mathbb{R}^{M\times2},
\end{equation}
the cost function (\ref{eq:cost}) can be rewritten as
\begin{equation}
\label{eq:costrew}
     V(\boldsymbol{\theta}) = [\boldsymbol{Y}-\boldsymbol{\Phi}\boldsymbol{\theta}]^T[\boldsymbol{Y}-\boldsymbol{\Phi}\boldsymbol{\theta}].
\end{equation}
By finding the least squares (LS) solution that minimises (\ref{eq:costrew})
\begin{equation}
    \boldsymbol{\hat{\theta}}_{LS} = [\hat{a}, \hat{b}]^T = (\boldsymbol{\Phi}^T\boldsymbol{\Phi})^{-1}\boldsymbol{\Phi}^T\boldsymbol{Y}.
\end{equation}
It is now possible to compute the estimate of the damping factor as
\begin{equation}
    \hat{\delta}=\frac{\hat{a}}{\sqrt{\pi^2 + \hat{a}}} 
\end{equation}
and of the frequency \(\omega\) as
\begin{equation}
    \hat{\omega} = \frac{1}{M} \sum_{k=0}^{M-1} \frac{\pi}{T_k}
\end{equation}
where \(T_k = t_{k+1}-t_k\) denote the measured
time interval between two consecutive peaks.
Finally an estimate of the natural frequency \(\omega_n\) is obtained as follows
\begin{equation}
    \hat{\omega}_n = \frac{\hat{\omega}}{\sqrt{1-\hat{\delta}^2}}.
\end{equation}
Once \(\hat{\omega}_n\) and \(\hat{\delta}\) have been estimated using the regression, the final step is to compute \(\hat{B}_b\) and \(\hat{k}\), using the nominal value \(J_b\):

 \begin{equation}
    \label{eq : params}
        \hat{B}_b = J_b\times(2\hat{\delta}\hat{\omega}_n), \qquad 
        \hat{k} = J_b\times(\hat{\omega}_n).
    \end{equation}

\subsection{Data sheet}
All the parameters used for the calculations are listed in the next listing:
\input{listings/global_params.tex}



